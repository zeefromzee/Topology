\documentclass[conference]{IEEEtran}

% Packages
\usepackage{amsmath,amssymb,amsthm}
\usepackage{graphicx}
\usepackage{cite}
\usepackage{url}
\usepackage{hyperref}

% Theorem environments
\newtheorem{theorem}{Theorem}
\newtheorem{lemma}{Lemma}
\newtheorem{proposition}{Proposition}
\newtheorem{definition}{Definition}

\begin{document}

\title{Topological Characterization of Access Structures in Secret Sharing via Simplicial Homology}

\author{
\IEEEauthorblockN{Authors}
\IEEEauthorblockA{Institution\\
Email: author@example.com}
}

\maketitle

\begin{abstract}
Secret sharing schemes are a foundational cryptographic primitive enabling the partitioning of a sensitive datum among a group of participants such that only predetermined authorized subsets can reconstruct the original secret. This paper presents a topology-based diagnostic framework for evaluating the feasibility, efficiency, and structural security of secret sharing access structures using tools from algebraic topology. We construct the unauthorized simplicial complex associated with an access structure, compute topological invariants including Betti numbers and Euler characteristic, and establish correlations between these invariants and cryptographic properties such as share size bounds and ideal scheme existence.
\end{abstract}

\begin{IEEEkeywords}
Secret sharing, access structures, simplicial homology, topological invariants, algebraic topology
\end{IEEEkeywords}

\section{Introduction}
\label{sec:introduction}

% To be completed

\section{Background}
\label{sec:background}

\subsection{Secret Sharing Schemes}
\label{subsec:secret-sharing}

% To be completed

\subsection{Access Structures}
\label{subsec:access-structures}

% To be completed

\subsection{Simplicial Homology}
\label{subsec:simplicial-homology}

% To be completed

\section{Methodology}
\label{sec:methodology}

\subsection{Unauthorized Complex Construction}
\label{subsec:complex-construction}

% To be completed

\subsection{Topological Invariants}
\label{subsec:topological-invariants}

% To be completed

\section{Theoretical Results}
\label{sec:theoretical-results}

% To be completed

\section{Computational Pipeline}
\label{sec:pipeline}

% To be completed

\section{Experimental Results}
\label{sec:experimental-results}

% To be completed

\section{Applications}
\label{sec:applications}

% To be completed

\section{Conclusion}
\label{sec:conclusion}

% To be completed

\section*{Acknowledgments}

% To be completed

\bibliographystyle{IEEEtran}
\bibliography{references}

\end{document}
